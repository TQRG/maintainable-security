\documentclass[11pt,fleqn]{article}

\usepackage{vmargin}
\setpapersize{A4}
\setmarginsrb{2.5cm}{2.5cm}{2.5cm}{2.5cm}%
{\baselineskip}{2\baselineskip}{baselineskip}{2\baselineskip}
\setlength{\parindent}{0pt}
\setlength{\parskip}{5pt}

\usepackage{caption}
\usepackage{subcaption}
\PassOptionsToPackage{hyphens}{url}\usepackage{hyperref}
\usepackage{epsfig}
\usepackage{latexsym}
\usepackage{url}
\usepackage{tgheros}
\renewcommand*\familydefault{\sfdefault}

\newcommand{\eline}{\vspace*{.75\baselineskip}}
\newcommand{\Referee}[1]{\eline \noindent {\bf Reviewer comment #1:} \\}
\newcommand{\Us}{\eline \noindent {\bf Response:}\\}
\newcommand{\TBD}{{\bf To Be Done}}
\newcommand{\newreviewer}[1]{\section*{Reviewer #1}\vspace*{-1.05\baselineskip}}
\newcommand{\editor}[1]{\section*{Editor #1}\vspace*{-1.05\baselineskip}}

%\usepackage{tikz}
\usepackage[skins,breakable]{tcolorbox}
\tcbset{textmarker/.style={
%
skin=enhancedmiddle jigsaw,breakable,parbox=false,before skip=-1mm, after skip=0mm,
boxrule=0mm,leftrule=2mm,rightrule=2mm,boxsep=0mm,arc=0mm,outer arc=0mm,
left=2mm,right=2mm,top=2mm,bottom=2mm,toptitle=1mm,bottomtitle=1mm,oversize}}

\newtcolorbox{rcomment}{textmarker,colback=gray!15!white,colframe=gray!80!white}

\newenvironment{revcomment}[1][]
{\Referee{#1}\begin{rcomment}}
{\end{rcomment}}

\newenvironment{reveditor}
{\begin{rcomment}}
{\end{rcomment}}

\usepackage{todo}
\usepackage{hyperref}

%customized
\usepackage{xfrac}

% more lenient line breaking (avoids text protruding into margins)
\sloppy

\title{\vspace*{-2cm}{\bf Authors' Response to the Review of
 EMSE-D-20-00300:\\
 ``Fixing Vulnerabilities Potentially Hinders Maintainability''}}

\author{Sofia Reis, Rui Abreu, Luis Cruz}
\date{}

\begin{document}

\maketitle

\editor{}

\begin{reveditor}
    Overall, I would like to give you a chance to respond to 
    these issues in a revision of your manuscript but I have 
    to make it clear that it does not guarantee a path towards 
    acceptance. Your rebuttal to \#1 will be of critical importance 
    for me judge the validity of the use of the BCH tool and in 
    making a final decision. Methodology is, of course, very 
    important for ESE journal and \#2 \& \#3s comments on methodological 
    details need to be carefully addressed, too. 
\end{reveditor}

\Us We thank the editor and reviewers for their valuable feedback. In the 
following, we address the concerns raised by the reviewers, and consequently 
to your concerns too.

\newreviewer{1}

\begin{revcomment}[1.1]

    This paper investigates the impact of patches to improve 
    security on the maintainability of open-source software. 
    The paper is well written and easy to read. I really 
    appreciated the replication package: complete, and well 
    described. Really a good work!
    
    However, I found a crucial issue in this paper. The authors 
    adopted as static analysis tool Better Code Hub. Better Code 
    Hub' model includes 10 guidelines that can help the developers to 
    write better code. Unfortunately, these guidelines are not related 
    to maintainability. Avoiding introducing the issues associated to 
    these guidelines do not imply increasing the maintainability of the 
    code, since the tools has no possibility to measure ``maintainability''.

\end{revcomment}

\Us Thank you for your positive feedback on the paper presentation and replication
package. 

We would like to rebut your concern about Better Code Hub (BCH)'s guidelines not 
being related to maintainability. First, as mentioned in the paper, BCH checks GitHub 
codebases against 10 engineering guidelines as devised by Software Improvement Group 
(SIG). These guidelines are the result of many years of experience: analyzing more 
than 15 million lines of code every week, SIG maintains the industry’s largest 
benchmark, containing more than 10 billion lines of code across 200+ technologies; SIG 
is, according to their website, the only lab in the world certified by TÜViT to issue ISO 25010:2011 
certificates for Trusted Product Maintainability\footnote{Information available here: 
\url{https://www.softwareimprovementgroup.com/methodologies/iso-iec-25010-2011-standard/}}.
TÜViT is an IT Security company. Being certified by TÜViT means 
that the company checked if BCH actually respects and follows the ISO 25010 guidelines.   

Second, the relation of the entire set of
guidelines to maintainability is well detailed in the O'Reilly's book ``Building Maintainable 
Software: Ten Guidelines for Future-Proof Code'' by Professor Joost Visser
\footnote{Available here: 
\url{https://www.softwareimprovementgroup.com/resources/ebook-building-maintainable-software/}}.
In fact, the underlying maintainability model for BCH is documented with several
(non-academic) reports linked at this page:
\url{https://www.softwareimprovementgroup.com/methodologies/iso-iec-25010-2011-standard/} as 
well as validated by the academic community in the following paper:

\begin{verbatim}
Heitlager, Ilja, Tobias Kuipers, and Joost Visser. 
``A practical model for measuring maintainability.'' 
6th international conference on the quality of information and 
        communications technology (QUATIC 2007). 
IEEE, 2007.
\end{verbatim}

BCH's compliance criterion is derived from the requirements for 4-star level maintainability (cf. ISO 25010). 
The concrete values of the thresholds used in BCH are also documented in their book Building Maintainability 
Software \url{https://www.softwareimprovementgroup.com/resources/ebook-building-maintainable-software/}.

Further, the methodology used to derive the thresholds is documented in the following 
literature:

\begin{verbatim}
    Tiago L. Alves, Christiaan Ypma and Joost Visser. 
    ``Deriving metric thresholds from benchmark data.'' 
    IEEE International Conference on Software Maintenance (ICSME 2010). 
    IEEE, 2010.

    Tiago L. Alves, Jose Pedro Correia and Joost Visser. 
    ``Benchmark-Based Aggregation of Metrics to Ratings.'' 
    2011 Joint Conference of the 21st International Workshop on Software 
    Measurement and the 6th International Conference on Software Process 
    and Product Measurement. 
    IEEE, 2011.

    Robert Baggen, Jose Pedro Correia, Katrin Schill and Joost Visser. 
    ``Standardized code quality benchmarking for improving software 
    maintainability.'' 
    Software Quality Journal.
    Springer, 2011.

\end{verbatim}

SIG performs the threshold calibration
yearly on a proprietary data set to satisfy the requirements of TUViT to
be a certified measurement model. However, please note that this data set 
cannot be shared due to non-disclosure agreements between SIG and its 
clients.

All in all, we argue that BCH's guidelines are a good proxy for building maintainable software. This is 
backed up by the observations of SIG in the field, as it was told us in personal communications: 

\begin{quote}
We saw the value of a tool like BCH in helping in writing clean and maintainable code and most importantly, 
it offers us so much more: over time, the guidelines in BCH will become part of your coding standards, and 
the overall quality of your work will improve, slowly, but surely.
\end{quote}

We changed the flow of the text and clarified these points in the \emph{Introduction}. We also 
added a section motivating Better Code Hub called \emph{Better Code Hub} in Section $3$.

\begin{revcomment}[1.2]

    As far as I know, the company that developed Better Code Hub 
    developed also the Delta Maintainability Model [diBiase2019] 
    that aims at measuring the maintainability of a code change 
    and a score and compare change-based maintainability measurements.

    [diBiase2019] M. di Biase and A. Rastogi and M. Bruntink and A. van 
    Deursen. The Delta Maintainability Model: Measuring Maintainability 
    of Fine-Grained Code Changes. IEEE/ACM International Conference 
    on Technical Debt (TechDebt) in 2019.

\end{revcomment}

\Us Thank you for suggesting Di Biase et al.'s paper. 
We evaluate the impact of security patches in software 
maintainability using an alternative maintainability model [Cruz2019]. 

Di Biase et al.'s work was developed in parallel with the one by Cruz et al [Cruz2019]. 
Both models attempt to measure maintainability, but Di Biase et al.'s is not as complete 
since it only considers $5$ of the same set of guidelines. Both models build their 
maintainability model over the SIG-MM model---the model behind BCH. 

Our model was validated by DiBiase in several discussions between 
the authors of both papers.

We also mention Di Biase et al.'s work in our \textit{Related Work} section:
\begin{quote}
    Recent work
proposed a new maintainability model to 
measure fine-grained code changes by adapting/extending the BCH model [diBiase2019].
Our work uses the same base model (SIG-MM), but considers a broader set of guidelines. 
Moreover, we solely focus on evaluating the impact of security patches on software maintainability.
\end{quote}

The literature that supports this response is the following:

\begin{verbatim}
    Luis Cruz, Rui Abreu, John Grundy, Li Li, Xin Xia. 
    ``Do Energy-oriented Changes Hinder Maintainability?'' 
    International Conference on Software Maintenance and 
    Evolution (ICSME 2019). 
    IEEE, 2019.

    M. di Biase and A. Rastogi and M. Bruntink and A. van 
    Deursen. 
    ``The Delta Maintainability Model: Measuring Maintainability 
    of Fine-Grained Code Changes.'' 
    IEEE/ACM International Conference on Technical Debt (TechDebt). 
    IEEE, 2019.

\end{verbatim}

\begin{revcomment}[1.3]

    Moreover, I am not sure that Better Code Hub is 
    the best static analysis tools to detect security 
    issues. One of the most adopted in security domain 
    is Coverity Scan for examples, but also other tools 
    might be a better choice (e.g. Kiuwan).

\end{revcomment}

\Us We do not think that the concern of the reviewer is warranted. 
Note that the goal of our work is not 
to use static analysis to detect security issues. We have already a dataset of 
security patches (pairs of vulnerable versions and non-vulnerable versions) and 
our objective is to understand what 
the impact of those patches on the maintainability
guidelines/metrics is. We argue that these guidelines/metrics
may in the future complement static analysis tools 
by assisting developers with more information on the 
risks associated with their patches.

\begin{revcomment}[1.4]

    The authors could have adopted both approaches, Better 
    Code Hub guidelines or another one first and then measure 
    the code maintainability with the Delta Maintainability Model 
    since the goal of the paper is to improve security on the 
    maintainability.

\end{revcomment}

\Us As we explain above, namely while addressing comment 1.2, both models assess 
maintainability. We consider the Delta Maintainability 
Model to be an alternative to our model. However, it only uses 5 guidelines of the 
set of guidelines our study considers. This has been 
discussed in the related work section.


\begin{revcomment}[1.5]

    In my opinion, the idea behind this work is very interesting, 
    but the research questions cannot be answered with the tools 
    and metrics selected. 

    My recommendation is to select a specific static analysis tool 
    for security issues and to include a maintainability model to 
    evaluate the maintenance.

\end{revcomment}

\Us Thank you for the positive feedback on the work. 

We would like to rebut your concern about the tools and 
metrics we are using. As we explained previously, in comment 1.1, 
BCH checks GitHub codebases against 10 maintainability guidelines 
as devised by Software Improvement Group (SIG). SIG has devised 
these guidelines after many years of experience: analyzing more 
than 15 million lines of code every week, SIG maintains the industry’s largest 
benchmark, containing more than 10 billion lines of code across 200+ technologies; SIG 
is the only lab in the world certified by TÜViT to issue ISO 25010:2011 
certificates\footnote{Information available here: 
\url{https://www.softwareimprovementgroup.com/methodologies/iso-iec-25010-2011-standard/}}.

The guidelines and their relation to maintainability is well 
detailed in the SIG's book ``Building Maintainable 
Software: Ten Guidelines for Future-Proof Code''
\footnote{Available here: \url{https://www.softwareimprovementgroup.com/resources/ebook-building-maintainable-software/}}.

Note that our goal is to measure the impact of security patches on software 
maintainability and not finding vulnerabilities---vulnerabilities
are already identified and located. Thus, for now, 
we do not need to leverage static analysis tools. 

For future research, it is indeed interesting to 
understand how BCH can be fused with a static analysis tool 
and how the BCH results can help 
security engineers assess the risk of their patches and guide 
them on performing better patches. But, for now, we only try
to understand the impact of security patches on those metrics.

All in all, we argue that BCH is appropriate 
to perform the current analysis.


\newreviewer{2}

\begin{revcomment}[2.1]

    Question:  Is your dataset available?

\end{revcomment}

\Us Thank you for your question. The dataset is fully available at 
\texttt{dataset/db\_security\_changes.csv} in the figshare package
provided in the contributions of the paper: 
\url{https://figshare.com/s/4861207064900dfb3372}. If the paper is 
accepted, the package will be available on GitHub.

This information is also available in Section 1 where we present 
the contributions of our work:

\begin{quote}
    This research performs the following main contributions:

\begin{itemize}
  \item Evidence that supports the trade-off between security and 
  maintainability: developers may be hindering software 
  maintainability while patching vulnerabilities.
	\item An empirical study on the impact of security patches on 
	software maintainability (per guideline, severity, weakness and 
	programming language).
	\item A replication package with the scripts and data created to 
	perform the empirical evaluation, for reproducibility. Available 
	online: \url{https://figshare.com/s/4861207064900dfb3372}.
\end{itemize}
    
\end{quote}


\begin{revcomment}[2.2]

    Abstract:
    Throughout the paper, "hypothesize" is probably a better word than 
    "suspect" for sounding more scientific.
    
    Your results should be more specific than "show evidence of trace-off". 
    Briefly tell us about what metrics you used and what the numerical results 
    indicate.
    
\end{revcomment}

\Us We thank the reviewer for the suggestions. We replaced ``suspect'' by
``hypothesize''. For the latter issue, we added to the abstract 
the results for maintainability and results
for the $2$ guidelines with a more negative impact on software 
maintainability (software complexity and unit size), in particular:

\begin{quote}
    Results show evidence of a trade-off 
between security and maintainability for $41.90\%$ of the cases, i.e., developers 
may hinder software maintainability. Our analysis shows that $38.29\%$ of patches increased
software complexity and $37.87\%$ of patches increased the percentage 
of LOCs per unit. 
\end{quote}

\begin{revcomment}[2.3]

    Intro:
    \begin{itemize}
        \item First sentence - quality is not ONLY related to cost but also to security 
        and safety.
        \item Provide a URL to Software Improvement Group and Better Code Hub.
        \item Page 3, Line 1 - Application Security Verification Standard (ASVS).
        \item Page 3, Line 7 - instead of a "broad number of code metrics" - tell 
        us exactly the number of metrics.
        \item Page 3, Line 13 - "suggest" sounds more scientific than "hint at"
        \item Page 3, Line 24 - "we intend to highlight the need …". Is that the 
        broad goal of your paper?  The goal should be explicitly stated in the 
        abstract and intro.
    \end{itemize}
\end{revcomment}

\Us Thank you for reporting these issues. All of them were addressed in 
the new version of the paper. In the last point, where the reviewer asks ``Is that the 
broad goal of your paper?'', we want to clarify that the goal of our study
is to show the impact of security patches on software maintainability and
highlight the need for solutions, which are described at the end of 
the abstract, Section 1 (\textit{Introduction}) and Section 5 (\textit{Study Implications}).

\begin{revcomment}[2.4]

    Intro:\\
    You should check out this paper: 
    [Li2017] Li, Frank and Paxson, Vern. A large-scale empirical study of security patches.
    ACM SIGSAC Conference on Computer and Communications Security in 2017.

\end{revcomment}

\Us Thank you for bringing our attention to this paper. We now discuss
this paper in the Related Work section and used it to support some 
of our claims and findings. 

We added the following paragraph to Section 7 (\emph{Related Work}):

\begin{quote}
    Researchers
performed a large-scale empirical study to understand the characteristics of security patches
and their differences against bug fixes [Li2017].
The main findings were that security patches are smaller and less complex compared
to bug fixes and usually performed at function-level. Our study compares
the impact of security patches on software maintainability with the impact 
of regular changes.
\end{quote}

We also used the paper to suport our research in different Sections of our paper: 

Section 1 (\emph{Introduction})
\begin{quote}
    Our hypothesis is that some of these patches may 
have a negative impact on the software maintainability and, 
possibly, even be the cause of the introduction of new 
vulnerabilities---harming software reliability and introducing 
technical debt. Research found that $34\%$ of the security patches 
performed introduce new problems and $52\%$ are incomplete and do not 
fully secure systems [Li2017].
\end{quote}

Section 3.2 (\emph{Security Patches vs. Regular Changes})
\begin{quote}
    Previous studies attempted to measure the impact of regular changes 
on open-source software maintainability. 
However, there is no previous work focused on comparing the impact 
of security patches with regular changes on maintainability, only
with bug-fixes [Li2017].
\end{quote}

Section 5 (\emph{Study Implications})
\begin{quote}
    \textit{\textbf{Prioritize High and Medium Severity:}} Previous research 
exhibits proof that developers prioritize
higher impact vulnerabilities [Li2017].
Our study shows that vulnerabilities of high and
medium severity should be prioritized in software maintainability tasks.
\end{quote}

\begin{revcomment}[2.5]

    Section 2\\
    \begin{itemize}
        \item The first paragraph is redundant with Section 1.  
        \item Page 4, Line 15 - You define the OCSP acronym late in the paragraph.
        \item Page 5, Line 6 - how many new branch points?
        \item Page 6, Line 19 - I think you want to say "In this study, maintainability is …."  
        Since it currently implies this information is available in the CWE.
        \item Page 6, line 47 - "regular commits" non-security commits to be more clear; maybe a 
        few more words to explain "randomly collected."  You also call them "baseline commits" in Figure 
        1 which gives a different phrase for the same concept.  Pick one and use it everywhere.
    \end{itemize}
\end{revcomment}

\Us Thank you for pointing out all of these issues. We addressed all of them. We decided
to use regular changes/regular commits instead of baseline and non-security commits.

\begin{revcomment}[2.6]
    Section 3\\
    \begin{itemize}
        \item Sometime you say 1330 patches (e.g. page 7, line 16) and sometimes 1300 (e.g. page 8, line 11, 
        and the abstract).  In science it's better to say the exact number and use it always and not round.  
        \item You say the dataset had 1330 (or 1300) patches and 1282 commits though you also say one patch 
        can have multiple commits assigned.  So how did you end up with less than one commit/patch?
    \end{itemize}
\end{revcomment}

\Us Thank you for raising this issue. The right number of security patches is $1300$ ($624$ from Ponta et al. and 
$676$ from Secbench). The $1330$ value is a typo, we fixed the issue. Regarding the second point, $1282$ commits is
the equivalent to the $624$ patches that integrate the Pontas et al. dataset where one patch
can have multiple commits. The $1282$ commits are referring to the Ponta et al. dataset and 
not to the combined dataset.

\begin{quote}
    We use a combined dataset of $1300$ security patches which is 
the outcome of mining and manually inspecting a total of $312$ 
GitHub projects.
\end{quote}

\begin{revcomment}[2.7]
    Section 3\\
    \begin{itemize}
        \item I would really like a better picture of what SIG is to establish credibility.  I was at first 
        thinking it had something to do with ACM SIG's but now after going to the website, it seems like a 
        commercial/consulting organization? I also feel irritated that I don't understand what you really mean 
        by "running against the BCH toolset" (page 8, line 13) since you have repeatedly mentioned BCH like it was 
        an industry standard - but you have not told me if it's a static analysis tool or what it is.  The explanation 
        and link to introduce SIG and BCH should be on Page 2 lines 41-42 rather than Page 7 - though an explanation 
        of the BCH toolset is best a paragraph in the methodology.  
    \end{itemize}
\end{revcomment}

\Us Thank you for raising the concern regarding SIG's credibility. SIG is a company that 
has more than $20$ years of experience and research in software quality production. 
Their models are scientifically proven and certified. We added this information and 
the links the reviewer refer in the comment to the Introduction section.

\begin{quote}
    As ISO does not provide any specific guidelines/formulas to 
calculate maintainability, we resort to Software Improvement Group 
(SIG\footnote{SIG's website: 
\url{https://www.sig.eu/} (Accessed on \today{})})'s web-based source 
code analysis service Better Code Hub (BCH)\footnote{BCH's 
website: \url{https://bettercodehub.com/} (Accessed on \today{})}  
to compute the software compliance with a set of $10$ 
guidelines/metrics to produce quality software based in ISO/IEC 
$25010$ [Visser2016]. SIG 
has been helping business and technology leaders drive their organizational 
objectives by fundamentally improving the health and security of 
their software applications for more than 20 years. Their 
models are scientifically proven and certified [Alves2010, Alves2011, Baggen2012].
\end{quote}

We rephrased "running against the BCH toolset" to "analyzed using the BCH toolset". We mean
that we used BCH to calculate the metrics presented on Table 1. 

The literature that supports this response is the following:

\begin{verbatim}
    J. Visser. 
    `` Building Maintainable Software, Java Edition: Ten Guidelines 
    for Future-Proof Code'' 
    O'Reilly Media, Inc. 2016.

    Tiago L. Alves, Christiaan Ypma and Joost Visser. 
    ``Deriving metric thresholds from benchmark data.'' 
    IEEE International Conference on Software Maintenance (ICSME 2010). 
    IEEE, 2010.

    Tiago L. Alves, Jose Pedro Correia and Joost Visser. 
    ``Benchmark-Based Aggregation of Metrics to Ratings.'' 
    2011 Joint Conference of the 21st International Workshop on Software 
    Measurement and the 6th International Conference on Software Process 
    and Product Measurement. 
    IEEE, 2011.

    Robert Baggen, Jose Pedro Correia, Katrin Schill and Joost Visser. 
    ``Standardized code quality benchmarking for improving software 
    maintainability.'' 
    Software Quality Journal.
    Springer, 2011.

\end{verbatim}


\begin{revcomment}[2.8]
    Section 3\\
    \begin{itemize}
        \item Page 7, line 4 - I don't think "resemble" is the word you are looking for - but I don't 
        know what you are trying to say so I can't make a suggestion.
    \end{itemize}
\end{revcomment}

\Us Thank you for raising this issue. In the sentence ``The codebases that resemble to the commits of our datasets 
are collected before the analysis performed by BCH.'', we mean that the codebases of each commit in 
our dataset were analyzed by BCH. We improved the text for:

\begin{quote}
BCH evaluates the codebase available in the default branch of a GitHub project. 
  We created a tool that pulls the codebase of each commit of our dataset 
  to a new branch; it sets the new branch as the default branch; and, runs 
  the BCH analysis on the codebase; after the analysis is finished, the tool saves 
  the BCH metrics results to a cache file.
\end{quote}

\begin{revcomment}[2.9]
    Section 3\\
    \begin{itemize}
        \item Please explain your repeatable methodology to "clean the dataset and select the most relevant and 
        compatible projects" (page 8, line 14) that brings the patches from 1282 to 969.  
        \item From 969 to 866 patches - 
        you just could not classify?  Please explain.
    \end{itemize}
\end{revcomment}

\Us As explained in comment $2.6$, the $1282$ commits refer only to the Ponta et al. dataset (a total of $624$ 
patches). In total, we analyzed $1300$ patches ($624$ patches from Ponta et al. and $676$ from Secbench). 
However, our study only considers $969$ patches. We detected a total of $23$ patches with floss-refactorings. 
In addition, BCH has some limitations, in particular, the lack of language support and project size,
and was uncapable of analyzing the rest of the $308$ data points. All of this is explained
in more detail later in Section 3.4. 

\begin{quote}
    Security patches can be performed through one commit (single-commit); 
    several consecutive commits (multi-commits); or, commit(s) interleaved with more 
    programming activities (floss-refactoring). Only $10.7\%$ of the data points 
    of our dataset involve more than one commit, the other
    $89.3\%$ of the cases are single-commit patches. We manually 
    inspected $25.1\%$ ($244$/$969$) of the security patches---$122$ from each dataset---and, 
    identified a total of $23$ floss-refactorings. Many of these 
    are patches with many changes where it is hard to understand which parts 
    involve the security patch. These were disregarded to minimize the impact 
    of measuring other programming activities rather than solely security patches
    can have in our results. 
    
    For projects with large codebases, the results calculated for 
    the \emph{Keep Your Codebase Small} guideline were way above the 
    limit set by BCH ($20$ Person-years). We suspect this threshold 
    may not be well calibrated. Thus, we decided not to consider 
    this guideline in our research. The 
    \emph{Automated Tests} guideline was also not considered since the 
    tool does not contemplate two of the most important techniques to 
    security testing: vulnerability scanning and penetration testing. 
    Instead, it only contemplates unit testing. In total, 
    we detected $308$ data points suffering from these two limitations. 
    Those data points were disregarded from the study.
\end{quote}

Those are the cases that were tossed from the initial $1300$ patches. 

Regarding the second question:
while manually inspecting the patches, we were not able to map
the issue to any CWE with confidence due to the lack of quality
information on the vulnerability/patch for $103$ patches. 
We clarified this in the paper.

\begin{quote}
    A total 
of $103$ patches were not classified because we were not able to map
the issue to any CWE with confidence due to the lack of quality
information on the vulnerability/patch.
\end{quote}

\begin{revcomment}[2.10]
    Section 3\\
    \begin{itemize}
        \item Page 8, line 33 "regular commit" - pick baseline or regular (or non-security) commits and use it always.  
        I don't understand the sentence "The baseline dataset is generated from the security commits dataset."   
        Do you mean you extract these commits from the projects in the security commits dataset?
    \end{itemize}
\end{revcomment}

\Us Yes, for each security commit we collected a random regular change from the same project. We have clarified it 
in the paper. 

\begin{quote}
    The baseline dataset is generated from the security commits dataset, i.e.,
for each security commit in the dataset, we collect a random regular
change from the same project. 
\end{quote}

\begin{revcomment}[2.11]
    Section 3\\
    \begin{itemize}
        \item Section 3.2 is confusing enough that I'm not sure of the analysis - 
        I will need to review the revision that you would submit to gain confidence.  
        Are you saying you want your non-security commits (regular commits) to be about 
        the same size as a given security commit so you are using this criteria to choose 
        your regular commits?  Justify this reasoning and make the methodology clear.  
        I would have expected that you randomly collected 1300 non-security commits to compare 
        with the 1300 security commits. Why constrain to make them be of "similar" size?  
        Why not allow the characteristics of regular versus security commits to be what 
        they are without such curating?
    \end{itemize}
\end{revcomment}

\Us Yes, we are looking for regular changes with the same size as security commits
from the same project. However, 
for some cases, it was difficult to find the exact same size. Thus, we  
searched for an approximation. Every 10 attempts of failling to get a commit with 
the same size, we spanned the range size. We clear the methodology in the paper
in Section 3.2 \textit{(Security Patches vs. Regular Changes)}.

``I would have expected that you randomly collected 1300 non-security commits to compare 
with the 1300 security commits.'': Although we started with 1300 commits, 
$331$ cases were tossed due to BCH limitations. In addition, BCH
takes substantial time to process this amount of cases. 
Since the comparison is between 
the final dataset of $969$ patches and the baselines, we produced 
baselines with the final number of viable results obtained for the security 
dataset.

We 
have collected a baseline of random changes only (no size restrictions) in the past and 
performed the same evaluation. We now present the results for both baselines: \textit{size-baseline}, where we consider
size appproximation; and, \textit{random-baseline}, where we consider all the regular
changes characteristics.


\begin{revcomment}[2.12]
    Section 3\\
    \begin{itemize}
        \item Since SIG seems to be "just another consulting organization" - I'd like 
        to see about the acceptance of the 10 guidelines in the software engineering 
        discipline.  I would guess you could find peer-reviewed references to support 
        these guidelines, or even books by Martin Fowler.   That has more credibility 
        than the reference by the SIG consulting group. For example on Page 9, line 42 
        you cite McCabe Complexity --  that has more credibility than a consulting 
        organizations guidelines that they use to make money on their tool.  Tell us more 
        about the BCH data experience since that is your comparison point for compliance.
    \end{itemize}
\end{revcomment}

\Us All the guidelines are fully described on their book ``Building Maintainable 
Software: Ten Guidelines for Future-Proof Code''
\footnote{Available here: \url{https://www.softwareimprovementgroup.com/resources/ebook-building-maintainable-software/}}.

We found other papers using some of the metrics measured by BCH. But we did not find
extra theoretical papers explaining metrics in detail as we found for McCabe Complexity.
However, metrics like Unit Size, Duplication, Unit Interfacing, Volume, Testability
and Code Smells are well-known metrics in the Software Engineering field. 

We would like to address your concern regarding the credibility of BCH. BCH checks GitHub 
codebases against 10 engineering guidelines as devised by Software Improvement Group 
(SIG). SIG has devised these guidelines after many years of experience: analyzing more 
than 15 million lines of code every week, SIG maintains the industry’s largest 
benchmark, containing more than 10 billion lines of code across 200+ technologies; SIG 
is the only lab in the world certified by TÜViT to issue ISO 25010:2011 
certificates\footnote{Information available here: 
\url{https://www.softwareimprovementgroup.com/methodologies/iso-iec-25010-2011-standard/}}.

The underlying maintainability model for BCH is documented with several
(non-academic) reports linked at this page:
\url{https://www.softwareimprovementgroup.com/methodologies/iso-iec-25010-2011-standard/} as 
well as validated by the academic community in the following paper:

\begin{verbatim}
Heitlager, Ilja, Tobias Kuipers, and Joost Visser. 
``A practical model for measuring maintainability.'' 
6th international conference on the quality of information and 
        communications technology (QUATIC 2007). 
IEEE, 2007.
\end{verbatim}

BCH's compliance criterion is derived from the requirements for 4-star level maintainability (cf. ISO 25010). 
The concrete values of the thresholds used in BCH are documented also in their book Building Maintainability 
Software \url{https://www.softwareimprovementgroup.com/resources/ebook-building-maintainable-software/}.

Further, the methodology for arriving at the thresholds is documented in the following 
literature:

\begin{verbatim}
    Tiago L. Alves, Christiaan Ypma and Joost Visser. 
    ``Deriving metric thresholds from benchmark data.'' 
    IEEE International Conference on Software Maintenance (ICSME 2010). 
    IEEE, 2010.

    Tiago L. Alves, Jose Pedro Correia and Joost Visser. 
    ``Benchmark-Based Aggregation of Metrics to Ratings.'' 
    2011 Joint Conference of the 21st International Workshop on Software 
    Measurement and the 6th International Conference on Software Process 
    and Product Measurement. 
    IEEE, 2011.

    Robert Baggen, Jose Pedro Correia, Katrin Schill and Joost Visser. 
    ``Standardized code quality benchmarking for improving software maintainability.'' 
    Software Quality Journal.
    Springer, 2011.

\end{verbatim}

SIG performs the threshold calibration
yearly on a proprietary data set to satisfy the requirements of TUViT to
be a certified measurement model. However, please note that this data set 
cannot be shared due to non-disclosure agreements between SIG and its 
clients. BCH has a free plan available that allows the analysis of public
projects with a size limit of 100.000 lines of code.


\begin{revcomment}[2.13]
    Section 3\\
    \begin{itemize}
        \item Page 10, line 49 "mainly single-commit patches" … "small percentage 
        of data points" - give us the exact numbers for each of these.  
        \item Page 11, line 38 - define "floss refactoring" and justify how these were objectively 
        and repeatably identified in your set. You say "these cases" which makes 
        it seems that you were including the set of patches that had more than one 
        commit as your (only) floss candidates.  
    \end{itemize}
\end{revcomment}

\Us Our dataset integrates $89.3\%$ of single-commit patches and 
$10.7\%$ of patches involving more than one commit. We replaced 
line 49 by the following sentence:

\begin{quote}
    Only $10.7\%$ of the data points 
of our dataset involve more than one commits, the other
$89.3\%$ of the cases are single-commit patches.
\end{quote}

We define floss-refactoring with the paragraph below. 
To mitigate the impact of 
floss-refactorings, we extracted and manually inspected a random sample with 
    $25\%$ of security patches from each dataset. From this sample, we identified 
    $23$ floss-refactorings. Most floss-refactoring patches include many changes 
    making it difficult to understand which parts involve the security patch. 
    Although we suspect that more floss-refactorings may occur, we argue that 
    they occur in a small portion of the data. We clarified 
this in the paper:

\begin{quote}
    Security patches can be performed through one commit (single-commit); 
    several consecutive commits (multi-commits); or, commit(s) interleaved with more 
    programming activities (floss-refactoring). Only $10.7\%$ of the data points 
    of our dataset involve more than one commit, the other
    $89.3\%$ of the cases are single-commit patches. To mitigate the impact of 
    floss-refactorings, we extracted and manually inspected a random sample with 
    $25\%$ of security patches from each dataset. From this sample, we identified 
    $23$ floss-refactorings. Most floss-refactoring patches include many changes 
    making it difficult to understand which parts involve the security patch. 
    Although we suspect that more floss-refactorings may occur, we argue that 
    they occur in a small portion of the data.
\end{quote}


\begin{revcomment}[2.14]
    Section 3\\
    \begin{itemize}
        \item Page 11, Line 42 - I think you are saying you removed these two guidelines 
        from the analysis of all projects (as shown in Figure 4) but this paragraph seems 
        to say this was only for projects with large code bases.
    \end{itemize}
\end{revcomment}

\Us We started with all the guidelines but after analyzing the BCH results, 
we found that some of the results for the \emph{Keep Your Codebase Small} guideline 
were way above the limit set by BCH ($20$ Person-years). We suspect this threshold 
may not be well calibrated. Thus, we decided not to consider 
this guideline in our research. We did not want 
to lose more data points. Thus, we decided to not consider the guideline for all projects.

We also did not consider the \emph{Automated Tests} guideline because
this guideline does not contemplate vulnerability scanning and penetration testing
which are the two most important testing techniques in security. 

We improved the paper in the following paragraph:

\begin{quote}
    Due to BCH limitations, in particular, lack of language support 
    and project size by BCH, $308$ data points were not analyzed and automatically 
    disregarded from our study. After performing the BCH analysis and 
    the maintainability calculations, we found the following 
    limitations regarding two of BCH's guidelines:
    
    1) For projects with large codebases, the results calculated for 
    the \emph{Keep Your Codebase Small} guideline were way above the 
    limit set by BCH ($20$ Person-years). We suspect this threshold 
    may not be well calibrated. Thus, we decided not to consider 
    this guideline in our research. 
    
    2) The 
    \emph{Automated Tests} guideline was also not considered since the 
    tool does not contemplate two of the most important techniques to 
    security testing: vulnerability scanning and penetration testing. 
    Instead, it only contemplates unit testing. 
\end{quote}

\begin{revcomment}[2.15]
    Section 3\\
    \begin{itemize}
        \item Page 11  line 48.  How many were disregarded?
    \end{itemize}
\end{revcomment}

\Us Due to BCH limitations, in particular, lack of language support 
and project size by BCH, $308$ data points were disregarded.
We clarified it in the paper (Section 3.4):

\begin{quotation}
    Due to BCH limitations, in particular, lack of language support 
    and project size by BCH, $308$ data points were not analyzed and automatically 
    disregarded from our study. 
\end{quotation}

\begin{revcomment}[2.16]
    Section 4\\
    \begin{itemize}
        \item Page 13, Line 3 - 969 security commits and 969 
        baseline commits?  It's hard to know how to parse that.
    \end{itemize}
\end{revcomment}

\Us We mean that our study is an empirical evaluation of $969$ security patches
and $969$ regular changes. We improved the sentence. 

\begin{quote}
    This study evaluates a total of $969$ security patches and 
    $969$ regular changes 
    from $260$ distinct open-source projects. 
\end{quote}

\begin{revcomment}[2.17]
    Section 4\\
    \begin{itemize}
        \item Page 13, Line 31 "overall patches …" rather than having us 
        rely on looking at the Figure, tell us the number or percentage of 
        when the impact is positive like you do with the negative in the next 
        sentence. [You do provide the specifics starting on page 15 line 49.  
        I wanted it here.]
    \end{itemize}
\end{revcomment}

\Us We replaced overall by the number of patches where maintainability 
increases ($38.7\%$).

\begin{quote}
    Regarding the impact of security patches per guideline, we 
observe that $38.7\%$ of the security patches have positive impact
on software maintainability.
\end{quote}

\begin{revcomment}[2.18]
    Section 4\\
    \begin{itemize}
        \item Page 19, Line 7 - for the unindoctrinated - I'd 
        explain the concept of Research Concepts. Also explain 
        your choice of 7a and 7b and how representative they are 
        from the several hundred choices. Similar for your explanations 
        in the second two paragraphs on this page.  How far down the 
        concept/vulnerability hierarchy tree are these CWE choices or 
        are they lower level vulnerabilities chosen from the 700+ CWE types?
    \end{itemize}
\end{revcomment}

\Us We added an explanation of the Research Concepts list to the paper
in a footnote:

\begin{quote}
    Research Concepts is a tree-view provided by the Common Weakness 
    Enumeration (CWE) website that intends to facilitate research into 
    weaknesses. It is organized 
    according to 
    abstractions of behaviors instead of how they can be detected, 
    their usual location in code, and when they are introduced in the 
    development life cycle. Available here: 
    \url{https: //cwe.mitre.org/data/definitions/1000.html}
\end{quote}

In Figure 7a, we present the entire distribution of the $969$ vulnerabilities per
each node of the 1st-level of the Research Concepts tree.
Looking at those results, we detected that the percentage of cases for CWE-707 ($30.4\%$) 
and CWE-664 ($32.8\%$) were considerably higher in the dataset than the rest of the CWEs.
Thus, we decided to look at lower levels of behavior for both CWEs: CWE-707
results are presented in Figure-7b and CWE-664 results are presented in Figure-7c.
In Figure-7b, all the CWEs presented are sub levels of the CWE-707. The same
happens for Figure-7c, but the CWEs are sub levels of the CWE-664.

We added the previous numbers to the paper and clarified that we are considering lower levels
of vulnerabilities.

\begin{quote}
    In \emph{RQ2}, we report/discuss the impact of security patches on
    software maintainability per weakness (CWE). We use the weakness definition
    and taxonomy proposed by the \emph{Common Weakness Enumeration} (cf. Section 2).
    Figure 7 shows three different charts. Figure \emph{7-a}, presents
    the impact of the $969$ patches grouped by the first level weaknesses from
    the \emph{Research Concepts} list. While the Figures \emph{7-b} and
    \emph{7-c} present the impact on maintainability for lower levels of 
    weaknesses for the most prevalent weaknesses in Figure \emph{7-a}:
    \emph{Improper Neutralization} (CWE-707) 
    and \emph{Improper Control of a Resource 
    Through its Lifetime} (CWE-664), respectively.
    
    In Figure~7-\emph{a}, there is no clear evidence of the impact on 
    maintainability per weakness. Yet, it is important to note that
    overall there is a very considerable number of cases that hinder
    maintainability---between $30\%$ and $60\%$. The CWE-707 and CWE-664 
    weaknesses integrate the higher number of cases compared to the remaining 
    ones: $295$ ($30.4\%$) data points and $318$ ($32.8\%$) data points, respectively. 
    Thus, we present an analysis of their sub-weaknesses on 
    Figure~7-\emph{b} and Figure~7-\emph{c}, respectively. 
\end{quote}

\begin{revcomment}[2.19]
    Section 4\\
    \begin{itemize}
        \item Page 20, Line 42 - how many is "a considerable number of cases"?  
    \end{itemize}
\end{revcomment}

\Us In total, we inspected around $25$ regular changes with no impact on software maintainability  
 to understand what was leading to the higher number of cases. This point
 was clarified in Section 4, with the following paragraph:

 \begin{quote}
    Overall, the results for both baselines, show that regular 
    changes are less prone to hinder the software 
    maintainability of open-source software.   
    However, the \emph{size-baseline} integrates a larger 
    number of cases with no impact on software maintainability. 
    We manually inspected a total of $25$ cases from
    that distribution of regular changes with no impact on maintainability, 
    and, found that identifying regular changes with the same 
    size as the security-related commit is limiting the type of regular 
    commits being randomly chosen: input 
    patches, variables or functions, type conversion (i.e., changes
    with no impact on the software metrics analyzed by BCH).
 \end{quote}

\begin{revcomment}[2.20]
    Section 4\\
    \begin{itemize}
        \item Page 20, Line 45 - it seems choosing commits of similar size 
        to the security commits ended up causing a limitation.  I'm still 
        wondering why you chose that methodology anyway.  
    \end{itemize}
\end{revcomment}

\Us As we described in comment $2.19$, we manually inspected $25$ cases.
Many of those cases were changes of small size ($< 10$ LOCs) where
refactorings were usually improving names of variables, functions, cases of type
conversion, etc. Changes that have no impact on the metrics we are 
evaluating. 

Our concern by presenting a total random baseline 
was that comparing regular changes with security changes with very different sizes---a scenario 
that could happen---was unfair. Thus, we chose to create this baseline.
However, we also curated a total random baseline in the past and we now 
present both baselines in the paper. 


\begin{revcomment}[2.21]
    Section 4\\
    \begin{itemize}
        \item RQ3 is answered to succinctly and with too little 
        detail to be convincing to me that the differences between 
        security and regular changes is different.  
    \end{itemize}
\end{revcomment}

\Us We improved the discussion of the comparison between 
the results for security patches and regular changes. We now
consider two baselines: \emph{size-baseline}, where 
we collected random changes with the same size as security 
patches; and, \emph{random-baseline}, where we collected
random changes without any restrictions. The overall results
show that regular changes are less prone to hinder software 
maintainability than security patches. Security
patches results have more cases with negative 
impact on software maintainability than any baseline 
of regular changes. By providing results 
for both baselines, we now clearly see that the limitation 
found for the \emph{size-baseline} does not change 
the need to pay more attention to security patches 
in software maintenance tasks.

\begin{revcomment}[2.22]
    Section 5\\
    \begin{itemize}
        \item Line 32 - since RQ3 was not convincing, 
        I don't feel a separate effort for maintainable security 
        is justified as separate from general focus on maintainability. 
    \end{itemize}
\end{revcomment}

\Us As explained in the previous response, security patches 
hinder more software maintainability than regular changes with size 
restriction (\emph{size-baseline}) or no size restriction (\emph{random-baseline}). Thus, we argue that
special attention should be given to maintainable security
when teaching software maintainability. We improved the analysis 
provided in Section 4 for the research question 3 (RQ3). We added 
an extra baseline for comparison with no size restrictions (\emph{random-baseline}) and 
improved the discussion over our results.


\begin{revcomment}[2.23]
    Section 6\\
    \begin{itemize}
        \item It might not be a threat - but I feel the curation 
        of the non-security commits is a limitation to the generalizability 
        of the characteristics of non-security commits
    \end{itemize}
\end{revcomment}

\Us We agree with the reviewer. Our concern by presenting a total random baseline 
was that 
comparing regular changes with security changes with very different sizes---a scenario 
that could happen---was unfair. 
Nevertheless, we also feel that considering size may be limitating the generalizability
of the characteristics of regular changes. Thus, we added to RQ3 our initial random 
baseline and completed the work by presenting now two baselines (size-baseline and random-baseline). 


\newreviewer{3}

\begin{revcomment}[3.1]

    The paper addresses a relevant and up to date topic. 
    The authors have analyzed a large set of project wrt to 
    quality assessment focused on vulnerabilities in line with 
    ISO25000 standard. There are some points worth considering: 
    - authors should better motivate the choice for BCH as web 
    based source code analysis service with respect to other 
    options that are available in literature. Tools such as Kiuwan, 
    sonar cloud, codify etc are also able to provide specific metrics 
    on maintainability and security, vulnerability features of code. 
    Furthermore Kiuwan is also an example designed based on the ISO25000 
    standard. Authors discuss the need for tools for risk assessment in 
    the study implications (section5) but fail to assess why their choice 
    falls on BCH. Furthermore all tools provide different conceptualization 
    and evaluation criteria for calculating quality characteristics such as Vulnerabilities 
    and Security. Authors should also explicit how these quality characteristics are measured in BCH. 

\end{revcomment}

\Us We kindly thank the reviewer for the comments. 

As mentioned in the paper, BCH checks GitHub codebases against $10$ engineering 
guidelines as devised by Software Improvement Group (SIG). SIG has devised 
these guidelines after many years of experience: analyzing more than $15$ million 
lines of code every week, SIG maintains the industry’s largest 
benchmark, containing more than $10$ billion lines of code across $200$+ 
technologies; SIG is the only lab in the world certified by TÜViT to issue ISO 25010:2011
certificates for Trusted Product Maintainability\footnote{Information available here: 
\url{https://www.softwareimprovementgroup.com/methodologies/iso-iec-25010-2011-standard/}}.

% Each of the guidelines in isolation is easy to understand, but, when 
% combined and evaluated as a whole, it becomes easy to see why 
%they would comprise the definition of good software. 
Furthermore, the relation of the entire set of
guidelines to maintainability is well detailed in the O'Reilly's book ``Building Maintainable 
Software: Ten Guidelines for Future-Proof Code'' by Professor Joost Visser
\footnote{Available here: 
\url{https://www.softwareimprovementgroup.com/resources/ebook-building-maintainable-software/}}.

In this study, to clarify, we only focus on measuring software maintainability, we do not focus 
on vulnerability detection or measuring security. Instead, we want to understand how software 
maintainability is impacted by security patches. However, as we explained before, BCH is 
based on the ISO 25010 which considers \emph{Security} as one of the main software product quality 
characteristics since $2011$.

There are other tools that measure the same metrics such as as Kiuwan; or, even 
provide a model to calculate maintainability such as SonarCloud. In this study, it is not
our goal to study other tools. However, we argue
that BCH is a better tool for the following reasons:

\begin{itemize}
    \item Kiuwan measures 4 guidelines that by name appear to be similar to 
    some of the guidelines measured by BCH: Complexity, Duplicated 
    Code, Size and Coupling\footnote{List available here: \url{https://www.kiuwan.com/docs/display/K5/Metrics}}. 
    However, we did not find the full description of kiuwan's metrics anywhere. Thus, 
    we are ensure that the metrics are the same but we are able to conclude that
    BCH integrates a larger set of metrics.  
    In contrast, 
    SIG provides a full description of the $10$ guidelines/metrics behind better code
    hub plus their relation with maintainability on the book and several scientific papers. 
    \item SonarCloud measures 5 similar metrics to the ones measured by BCH: Complexity, Duplication, 
    Issues, Size and Tests\footnote{List available here: \url{https://sonarcloud.io/documentation/user-guide/metric-definitions/}}. 
    The tool rates maintainability with a letter based on the technical 
    debt ratio which is calculated as the ratio between the cost to develop 
    the software and the cost to fix it. But no information is provided 
    on how the remediation cost and developmet cost are calculated. 
    Better Code Hub contemplates a larger set of maintainability metrics 
    and our formula for maintainability is not only fully explained in our paper but also 
    based on metrics that are fully and publicly described.  
\end{itemize}

We do not know codify and did not find the tool anywhere in static analysis tools 
collections or even through an organic search at Google.  

We added a Section to 
the paper called \textit{3.3 Better Code Hub} where we motivate the Better Code Hub tool and 
compare it with Kiuwan and SonarCloud.


\begin{revcomment}[3.2]

    In section 3a better state/formulate the hypothesis authors 
    analyze wrt to the Research questions in previous sections. 
    In general the empirical study is not structured according to any 
    of the guidelines or explicit states any of them. 

\end{revcomment}

\Us In our work, we follow the same methodology as the previous work 
mentioned below but instead of measuring
the impact of energy-oriented changes on software maintainability, 
we measure the impact of security patches.

\begin{verbatim}
    Luis Cruz, Rui Abreu, John Grundy, Li Li and Xin Xia. 
    ``Do Energy-oriented Changes Hinder Maintainability?.'' 
    IEEE International Conference on Software Maintenance and 
    Evolution (ICSME). 
    IEEE, 2019.
\end{verbatim}

Improving software security is not a trivial task and 
requires implementing patches that might affect software 
maintainability. Our hypothesis is that some of these patches may 
have a negative impact on the software maintainability. With this 
study, we want to understand which maintainability metrics can guide 
security engineers on producing patches with better quality---which 
we evaluate with \textit{\textbf{RQ1: What is the impact of security patches on the
maintainability of open-source software?}}. In this research 
question, we present the results by metric, overall score, severity and programming language
to have a better understanding of the impact of each of these parameters in 
software maintainability. 

Different weaknesses may require patches with different characteristics. 
Our hypothesis
is that security patches for different weaknesses can have different 
impacts on software maintainability. Thus, we evaluate this in \textit{\textbf{RQ2: Which weaknesses are more likely to
affect open-source software maintainability?}}. The results can help
security engineers prioritizing weaknesses and bring awareness to the 
ones that need more attention.

Performing a regular change/refactoring, for instance, to improve the name of 
a variable or function, is different than performing a security patch.
Therefore, we created \textit{\textbf{RQ3: What is the impact of security patches versus 
regular changes on the maintainability of open-source software?}}
to explore if there is a need to pay more attention to 
security patches than regular changes during software maintenance tasks.

The previous information is already included in the paper. However, 
we clarified these points in Section 2, where we present the research questions 
and our hypothesis. 


\begin{revcomment}[3.3]

    The findings of the research should better emphasize what the 
    lessons learned and what better practices should be put in place 
    by developers to assure better quality of software. The take away 
    message remains hindered. 

\end{revcomment}

\Us We agree with the reviewer that we should better emphasize the lessons 
learned and what best practices can be followed by developers to assure better 
vulnerability patches. We improved the discussion of these points in Section 5.

\end{document}
\documentclass[11pt,fleqn]{article}

\usepackage{vmargin}
\setpapersize{A4}
\setmarginsrb{2.5cm}{2.5cm}{2.5cm}{2.5cm}%
{\baselineskip}{2\baselineskip}{baselineskip}{2\baselineskip}
\setlength{\parindent}{0pt}
\setlength{\parskip}{5pt}

\usepackage{caption}
\usepackage{subcaption}


\usepackage{epsfig}
\usepackage{latexsym}
\usepackage{url}
\usepackage{tgheros}
\renewcommand*\familydefault{\sfdefault}

\newcommand{\eline}{\vspace*{.75\baselineskip}}
\newcommand{\Referee}[1]{\eline \noindent {\bf Reviewer comment #1:} \\}
\newcommand{\Us}{\eline \noindent {\bf Response:}\\}
\newcommand{\TBD}{{\bf To Be Done}}
\newcommand{\newreviewer}[1]{\section*{Reviewer #1}\vspace*{-1.05\baselineskip}}
\newcommand{\editor}[1]{\section*{Editor #1}\vspace*{-1.05\baselineskip}}

%\usepackage{tikz}
\usepackage[skins,breakable]{tcolorbox}
\tcbset{textmarker/.style={
%
skin=enhancedmiddle jigsaw,breakable,parbox=false,before skip=-1mm, after skip=0mm,
boxrule=0mm,leftrule=2mm,rightrule=2mm,boxsep=0mm,arc=0mm,outer arc=0mm,
left=2mm,right=2mm,top=2mm,bottom=2mm,toptitle=1mm,bottomtitle=1mm,oversize}}

\newtcolorbox{rcomment}{textmarker,colback=gray!15!white,colframe=gray!80!white}

\newenvironment{revcomment}[1][]
{\Referee{#1}\begin{rcomment}}
{\end{rcomment}}

\newenvironment{reveditor}
{\begin{rcomment}}
{\end{rcomment}}

\usepackage{todo}
\usepackage{hyperref}

%customized
\usepackage{xfrac}

% more lenient line breaking (avoids text protruding into margins)
\sloppy

\title{\vspace*{-2cm}{\bf Authors' Response to the Review of
 EMSE-D-20-00300:\\
 ``Fixing Vulnerabilities Potentially Hinders Maintainability''}}

\author{Sofia Reis, Rui Abreu, Luis Cruz}
\date{}

\begin{document}

\maketitle

\editor{}

\begin{reveditor}
    Overall, I would like to give you a chance to respond to 
    these issues in a revision of your manuscript but I have 
    to make it clear that it does not guarantee a path towards 
    acceptance. Your rebuttal to \#1 will be of critical importance 
    for me judge the validity of the use of the BCH tool and in 
    making a final decision. Methodology is, of course, very 
    important for ESE journal and \#2 \& \#3s comments on methodological 
    details need to be carefully addressed, too. 
\end{reveditor}

\Us We thank the editor and reviewers for their valuable feedback. In the 
following points, we address the concerns raised by the reviewers.

\newreviewer{1}

\begin{revcomment}[1.1]

    This paper investigates the impact of patches to improve 
    security on the maintainability of open-source software. 
    The paper is well written and easy to read. I really 
    appreciated the replication package: complete, and well 
    described. Really a good work!
    
    However, I found a crucial issue in this paper. The authors 
    adopted as static analysis tool Better Code Hub. Better Code 
    Hub' model includes 10 guidelines that can help the developers to 
    write better code. Unfortunately, these guidelines are not related 
    to maintainability. Avoiding introducing the issues associated to 
    these guidelines do not imply increasing the maintainability of the 
    code, since the tools has no possibility to measure "maintainability".

\end{revcomment}

\Us Thank you for your positive feedback on the paper presentation and replication
package. BCH is not a static analysis tool instead it checks codebases for 
compliance with the ten guidelines. The set of guidelines presented in our paper
is the same published by Software Improvement Group (SIG) in their ebook called
"Building Maintainable Software: Ten Guidelines for Future-Proof Code"
\footnote{Available here: https://www.softwareimprovementgroup.com/resources/ebook-building-maintainable-software/}.
According to the authors, all the guidelines were infered from analyzing
hundres of real-world systems and measure maintainability.

We clarified the difference between BCH and static analysis tools in
the Section 1 (Introduction) and Section 3 (Methodology).

\begin{revcomment}[1.2]

    As far as I know, the company that developed Better Code Hub 
    developed also the Delta Maintainability Model [diBiase2019] 
    that aims at measuring the maintainability of a code change 
    and a score and compare change-based maintainability measurements.

    [diBiase2019] M. di Biase and A. Rastogi and M. Bruntink and A. van 
    Deursen. The Delta Maintainability Model: Measuring Maintainability 
    of Fine-Grained Code Changes. IEEE/ACM International Conference 
    on Technical Debt (TechDebt) in 2019.

\end{revcomment}

\Us Thank you for suggesting Marco's paper. Our paper uses a model 
for measuring maintainability that was published previously by one of the co-authors
of this paper [Cruz2019]. The entire methodology and calculations 
were actually validated by Marco. 

The model created by Marco focus on measuring maintainability of fine-grained 
changes using 5 of the same guidelines we use. The SIG-MM model in 
Marco's paper is the model behind Better Code Hub. 

We added Marco's work to our related work. 

[Cruz2019] Luis Cruz, Rui Abreu, John Grundy, Li Li, Xin Xia. 
Do Energy-oriented Changes Hinder Maintainability?. International 
Conference on Software Maintenance and Evolution (ICSME) in 2019.


\begin{revcomment}[1.3]

    Moreover, I am not sure that Better Code Hub is 
    the best static analysis tools to detect security 
    issues. One of the most adopted in security domain 
    is Coverity Scan for examples, but also other tools 
    might be a better choice (e.g. Kiuwan).

\end{revcomment}

\Us Thank your for your comment. In this paper, we do not
try to use static analysis to detect security issues.
We have already a dataset of security patches (vulnerable versions
and safe versions) and our goal is to understand what 
is the impact of those patches on the maintainability
guidelines/metrics. We arguee that these guidelines/metrics
can in the future complement static analysis tools 
by assiting developers with more information on the 
risks associated with their patches.

SonarQube uses 4 of the set of metrics we are using and 
also rates maintainability. In Section X, 
we compare the results between BCH and SonarQube. However, 
we arguee that BCH performs a more complete analysis.

\begin{revcomment}[1.4]

    The authors could have adopted both approaches, Better 
    Code Hub guidelines or another one first and then measure 
    the code maintainability with the Delta Maintainability Model 
    since the goal of the paper is to improve security on the 
    maintainability.

\end{revcomment}

\Us Thank you for your comment. As we explained before, while 
addressing comment 1.2, both models assess maintainability.
Delta Maintainability Model is an alternative to our approach 
that uses only 5 of the set of guidelines our study considers.

\begin{revcomment}[1.5]

    In my opinion, the idea behind this work is very interesting, 
    but the research questions cannot be answered with the tools 
    and metrics selected. 

    My recommendation is to select a specific static analysis tool 
    for security issues and to include a maintainability model to 
    evaluate the maintenance. 

\end{revcomment}

\Us Thank you for the positive interest on the work. As we 
explained before Better Code Hub calculates maintainability metrics.
Our goal was to measure the impact of security patches on software 
maintainability. For future research, it would be interesting to 
understand if the results of these metrics can actually help 
security engineers assessing the risk of their patches and guide 
them on performing better patches. However, in this work, 
we only focus on assessing software maintainability.

\newreviewer{2}

\begin{revcomment}[2.1]

    Question:  Is your dataset available?

\end{revcomment}

\Us Thank you for your question. The dataset is fully available at 
\texttt{dataset/db\_release\_security\_fixes.csv} in the figshare package
provided in the contributions of the paper: 
\url{https://figshare.com/s/4861207064900dfb3372}. If the paper is 
accepted, the package will be available on GitHub. 

\begin{revcomment}[2.2]

    Abstract:
    Throughout the paper, "hypothesize" is probably a better word than 
    "suspect" for sounding more scientific.
    
    Your results should be more specific than "show evidence of trace-off". 
    Briefly tell us about what metrics you used and what the numerical results 
    indicate.
    
\end{revcomment}

\Us We thank the reviewer for the suggestions. We addressed
both of the issues in the abstract. For the later issue, we reported 
the overall result for maintainability and results
for the two of the guidelines with more negative impact in software 
maintainability: software complexity and unit size.

\end{document}
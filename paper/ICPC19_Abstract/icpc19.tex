\documentclass[a4paper]{article} %
\usepackage{graphicx,amssymb} %

\textwidth=15cm \hoffset=-1.2cm %
\textheight=25cm \voffset=-2cm %


\date{\today} 

\def\keywords#1{\begin{center}{\bf Keywords}\\{#1}\end{center}} %

\begin{document}

\title{Security vs. Maintainability: \\ Fixing Vulnerabilities Obfuscates your Code}

\author{Anonymou(s) Author(s)}


\maketitle

\thispagestyle{empty}

\begin{abstract}

Nowadays, security is an absolutely necessary requirement for the production of quality software. However, there is still a considerable amount of zero-day vulnerabilities published and fixed almost every week. The authors suspect that when developers address these vulnerabilities on their codebases, they actually affect the maintainability of software and even introduce more vulnerabilities. This paper evaluates the impact of refactorings to improve security on the maintainability of open-source software. Maintainability is measured based on the \emph{Better Code Hub}'s model of 10 guidelines for both before and after security commits for a dataset including 607 security commits. Results show that fixing software vulnerabilities harms the maintainability of open-source software. Specially for refactorings that deal with \emph{Broken Authentication and Session Management}, \emph{Memory Leaks} and \emph{Denial-of-Service} attacks. Therefore, it is necessary to bring awareness for coding good practices and for the use of automated tools that have the potential to help developers with this issue.  
 
\end{abstract}

\keywords{Security, Software Maintenance, Open-Source Software} 






% \section{Introduction}






\end{document}
